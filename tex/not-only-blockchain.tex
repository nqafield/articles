\documentclass{main}
\title{Blockchain Was the First, But Will Not Remain the Only}
\begin{document}
\begin{abstract}
One of the most visible technical innovation of the last ten years,
aside from the promise to get tourists on Mars, is Blockchain. However,
due to the large amount of marketing noise and financial speculations, many
of us missed the technical trend this technology has successfully started.
The trend, which can be labeled as Zero-Trust Decentralization (ZTD), and
which is truly unique and new. What is the future of it and should we
expect other products aside from Blockchain?
\end{abstract}

Decentralized systems are not new at all. Take, for example, Torrent, a file sharing
protocol. It doesn't have a cental place of control, each node
works indendently and anonymously, sending parts of the data to other nodes
and receiving other parts back.  However, despite our inability to trust
any individual node, the receiver of a file has a guarantee that it is
exactly the file that was initially injected into the network by its sender.
This guarantee is pretty easy to achieve just by checking hash-codes
of file parts and verifying them later at the time of receiving.

Technically speaking, Torrent is a single-transaction system, where the
receiver and the sender trust each other. They only don't trust the system,
that's why they use checksums to make sure there were no data corruption
on the way.

A bigger problem arrises when the receiver doesn't trust the system \emph{and}
the sender. A digital money sending business case is a perfect example. If money
is sent via, say, \href{https://www.paypal.com}{PayPal}, the receiver doesn't trust the sender, but it
does trust the system. The centralized database of PayPal Inc. guarantees
the seller of a book that the payment with its unique ID will
not disappear from the database after the book is shipped.

Is it possible to create an architecture, where both the sender and the system
will be anonymous and untrustable, but the combination of all participants will
make the entire system trustable, to some extent? Blockchain was the first
solution which suggested that it was indeed possible.

In Blockchain, each node maintains the entire ledger of transactions
connected sequentially and signed with expensive hash codes. In order to modify
a transaction in the middle of the ledger a sender would have to recalculate
all hash codes coming after it, which is, the longer the tail of the ledger,
the more time consuming operation. On ther other hand,
if such a recalculation won't be done and the sender just changes the
transaction in the middle of the ledger, other nodes won't believe it and
will reject the modification. That's how their software is designed to behave
and they all have the same version of the software on-board.

This seems to be a solid concept, which proved itself in a number of real-life
applications, including Bitcoin and many other Blockchain-based cryptocurrencies.
However, this is just the beginning. This is the first, but not the only
possible solution in the terrory of Zero-Trust Decentralization. There
are others, for example, based on Decentralized Acyclic Graph (used in
IOTA and ByteBall cryptocurrencies, for example).

\href{https://www.zold.io}{Zold}, yet another solution, which was invented and
developed by the author, shifts
the decision of trust to the client's side, allowing money receiving
software decide, which node has a more trustable version of the data. The data
coming from the majority of nodes will be considered as more trustable.
A more detailed explanation of the architecture can be found in the
\href{https://papers.zold.io/wp.pdf}{White Paper}.

It is only reasonable to believe that there will more solutions to solve
the same problem. They will use different principles of data organizations,
to guarantee trust, and will have their own advantages and drawbacks. One of the
most visible disadvantage of Blockchain, for example, is its speed of transaction processing.
Obviously, since all transactions must be inserted
sequentially into the same ledger, it's impossible to achieve high speed
of processing, unless some additional optimization is applied
(Bitcoin lighting network and side-chains are just a few examples).

Thus, the future of Zero-Trust Decentralization is indeed full of innovative
and interesting solutions. However, our current primary focus on Blockchain is
a potential threat to the development of those innovations. We must
admit that Blockchain was a great concept, which opened the door to the
new technical domain, but it's time to start researching and finding
better or just different alternatives.

\end{document}
