\documentclass{main}
\title{A Software Architect Is the Person You Blame}
\begin{document}
\begin{abstract}
How efficient is your current software project, and could it potentially benefit
from the addition of a software architect? More importantly, what exactly
does a software architect do and what can they provide to your team?
With the world of software development rapidly moving towards more agile workflows
amidst democracy in the front seat, the importance of the software architect
is understated. A position misunderstood by many is a crucial component that delivers
unparalleled guidance in the project pipeline, assigning responsibility, to an
individual who can turn a company vision into code.
\end{abstract}

Some might believe that the title of a software architect is merely a status
symbol placed upon a senior coder, signaling that a specific level of respect
should be delivered; this assumption is wrong. The job of the architect is one
that can be highly significant if it is adequately bestowed and the person who
receives the title has the qualifications necessary to lead a team. Most
importantly, the individual must be able to take the blame for project failures.

The software architect is the individual who takes the blame for when a project
fails or is praised when the software, and the team, succeeds. Now, we must
understand what is meant when using the word ``blame'' and why such a large
association would be placed with an individual. The software architect is your
team's guide; they are selected to carry the initial vision to a fully
solidified working piece of code. As leaders, they elect to take the responsibly
for the direction in which they lead their team.

Lead Software Engineer at EPAM Systems, Nikolay Ashanin, compared the
responsibility of a software architect to that of a bridge worker from the 19th
century in his published article
\href{https://medium.com/@nvashanin/the-path-to-becoming-a-software-architect-de53f1cb310a}{The Path to Becoming a Software Architect} and
had said at that time that the key group of engineers, architects and workers
stood under the bridge while the first vehicles were on it; thus, they staked
their lives upon the construction and the strength of the structure.

When we say that a software architect must absorb the blame for a project, we
are merely saying that the project outcome that is produced shall fall upon
their shoulders. It is entirely up to the software architect to delegate
responsibilities of a project utilizing their methodologies whether that be
additional toolsets, their authority, or mentorship and coaching.

Project managers do not always have the option to hire a software architect, as
they are typically individuals who are curated by their company, learning and
understanding their team over time. In an
\href{https://www.infoq.com/articles/architecture-five-things}{excellent article} by Simon Brown of
InfoQ, a division of C4 media that focuses on software development, he noted
that ``becoming a software architect isn't something that happens overnight or
with a promotion. It's a role, not a rank.''

Most importantly, the decision of a software architect must be treated as final.
Otherwise, without a true final say in the matter, the individual won't be
looked upon as an authoritative figure. Even a project manager must treat the
software architect as the final decision maker when it comes to implementing and
producing code. Rather than overruling the decisions of their architect, project
managers should seek to replace the individual if product end visions are not
adequately aligning. An individual does not need to be fired, but perhaps placed
back within the standard pool of programmers; over time, they might
professionally grow to attempt the opportunity once more.

A software architect is the guiding rails for a project; they keep their team of
developers moving forward and on-vision while accepting the responsibilities for
the team's actions as a whole. Not only must an architect be able to lead, but
also to understand the skills of their team, and how they can contribute to a
finished project.

Beyond the ability to craft beautiful code, lead a team to completion, and work
under pressure, a software architect must stand as a figure able to accept
responsibility for a project; this is the characteristic that defines a true
architect. More than simply a senior programmer, more than simply a leader, the
software architect stands as a gatekeeper for quality and as a guiding vision for
their team. In the end, whether the result is positive or negative, the software
architect can stand up and take the praise or blame for what their team
accomplished.

\end{document}
