\documentclass{main}
\title{Creative Programmers May Be Dangerous}
\begin{document}
\begin{abstract}
Letting programmers use each and every possible technique and design pattern may lead to low quality of code;
that's why it's desirable to keep their imagination under strict control.
\end{abstract}

``Another lesson we should
have learned from the recent past is
that the development of `richer' or
`more powerful' programming languages
was a mistake in the sense
that these baroque monstrosities,
these conglomerations of idiosyncrasies,
are really unmanageable, both
mechanically and mentally,'' said \textcite{dijkstra1972}.

``The more unconstrained degrees of freedom there are in the software
development process, the more opportunity there will be for
unexpected outcomes... Every programmer should produce essentially the
same code for the same design.'' said \textcite{munson2003}.

Expressiveness (or expressive power) was traditionally considered a virtue of programming languages.
Some of the examples are:

Syntactic sugar. Most modern programming languages are proud of syntax
  features that make code more compact and elegant.

Domain Specific Languages. Libraries like Nokogiri build their own
  syntax layer on top of the existing programming language, which is more
  compact and expressive.

Default arguments. In some programming languages, like Ruby and PHP,
  it's possible to free programmers from the necessity to specify
  all arguments of a function when calling it; omitted parameters will
  be assigned to default values.

Lose typing. Weak and dynamic typing (like JavaScript and Ruby)
  make code shorter, comparing to strong and static typing (like C++ and Java).

Larger monitors. Both hardware and software allow to display bigger
  pieces of code, both horizontally and vertically, helping programmers
  be much freer in their code writing and formatting decisions.

However, static analyzers are going in the opposite direction,
making many expressive practices illegal.
For example, tternary boolean operator and
postfix increment operator ++ are prohibited by Checkstyle in Java, even
though the former one turns a six-lines code block into a one-liner---they
both are considered too confusing for the reader.

Each programming language has its own set of static analyzers: ...

Quality of code is a very important component of the overall product quality.

Readability of code is the key contributor to its quality.

It seems that the more expressive programmers are, the lower the readability.

Mostly because all programmers have their own unique understandings of what is right and how the code should look like.

Hence, uniformity of code makes it more readable to a bigger amount of people.

Thus, the more creative and imaginative is the programmer, the lower the quality of code he or she may produce, if not being restricted.

\end{document}
