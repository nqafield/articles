\documentclass{main}
\title{Breaking the Chains of Freelance Slavery}
\begin{document}
\begin{abstract}
Ten years ago, spammer scams claiming you could earn \$85,000
dollars per year all while working from home were all the rage.
The idea was mind blowing for the average worker bee. Sit at home
earning a solid wage? Imagine all that ``convenience.'' A dream come true.
These scammers are still around, but the idea of working from home is no longer so novel.
\end{abstract}

In fact, 70 percent of people worldwide work remotely at least once a week,
although most still need to show up at the office regularly. Still, at least 3.9
million people in the United States work from home at least half the week. For
many workers, this may sound like great news. They don't have to deal with long
commutes,   difficult personalities around the office, or the soul crushing
environment of a cube farm.

Still, there are drawbacks to consider. For one, the increase in remote work
also coincides with an increase in the use of contractors who are paid an hourly
wage but often aren't provided with benefits, such as healthcare and retirement
contributions. These days, roughly 20 percent of the American workforce works
under a contract rather than full-time, and many of them are remote workers.

Even outside of the benefits issue, remote work comes with many hassles. You
might think you're setting yourself free, but quite often your putting yourself
in prison, and your jail happens to be your home.

In the office, you might worry about a boss looking over your shoulder, or
security cameras monitoring you and your productivity. A company's not going to
install a camera at home though, right? Probably not, but that's because they
don't have to. Your employer can simply install monitoring software on your
computer to track what you're doing.

Yuck. And it gets worse. Remote work means fewer meetings, right? Most people
don't want to sit in a conference room, listening to so-and-so drone on about
such-and-such. Problem is, you can still end up stuck in a virtual meeting, with
so-and-so droning on about such-and-such anyways. Except this time around he or
she is simply talking through a mic, perhaps with a video feed playing in a
window on your screen.

But hey, outside of the meetings and the tracking software, you have your
freedom, right? The day is still yours. You can simply work away and get things
done without all the ``chatter'' of the office. Ha! Say hello to Slack and other
instant messaging platforms. Instead of dealing with the guy or gal in the next
cubicle, you have to deal with messages from countless teammates pouring in.

And we're not done yet. In my experience I've observed another important trend:
employers tend to respect your working hours less and less when you work
remotely. When you clock out of the office, you cut that tether with your
company. Sure, your boss might email or call you, but a decent employer will
still recognize that you're off the clock

When you work remotely, it can be hard for your boss to separate your working
hours from your personal life. After all, when you're a remote worker and you're
at home, you're still ``in the office.'' So why can't you finish up such and such
project tonight? Is it really unreasonable to expect you to answer emails on
Sunday morning? After all, you're still in the office.

The answer, of course, is yes, that's unreasonable. Your personal time is your
personal time. Yet many employers don't see it that way with remote workers. As
a result, many remote workers end up putting in overtime, often unpaid.

Further, remote workers have to deal with their own brand of office politics.
Maybe you don't have to deal with an annoying coworker 40 hours a week, but you
may have to deal with paranoid managers and coworkers.

How does the team know you're not watching TV or playing video games? Sure,
there's tracking software, but that only monitors your screen, not what's going
on around you. You might have your computer open, plugging away at work, while
Netflixing ``the Office'' and laughing at all the Office shenanigans.

When you're working in an actual office, people can easily verify that you're
not wasting company time. They can quite simply walk by your desk and see what
you're up to. You're under constant surveillance but that does have some
benefits, including the simple fact that your boss can see what you're doing.

There are productivity issues as well. What happens if you're computer suffers a
failure? If you're at home, you can't simply walk it down to the IT department.
You either have to buckle up for a commute or solve the problem on your own.

What if you need to find a piece of vital information or a specific form? When
you're at home, you can't simply ask someone around the office. You can send an
instant message, of course, but don't expect instant replies. Instead, you find
yourself digging through manuals, online document dumps, and wikis.

This doesn't mean that working remotely is hell on Earth, but it's not heaven
either. What's important is setting clear expectations and knowing the hassles
before you sign up for working remotely. If you know what you're getting into,
you may be able to mitigate the many challenges.

For many workers, however, an old fashioned office works best. A regular
schedule means regular hours. And while working with teammates and supervisors
has hassles, it also offers many benefits. Maybe remote work is right for you.
However, don't assume that to be true simply because you want to work from the
comforts of your own home.

Fact is, your office is where the work is. Rather than enjoying the freedom of
being at home, you might find yourself struggling to separate your private and
professional lives.

\end{document}
