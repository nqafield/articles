\documentclass{main}
\title{The Era of Hackers Is Over}
\begin{document}
\begin{abstract}
The ability to hack algorithms and make data structures optimal
was a virtue of programmers a few decades ago. However, the current reality is
much different with the ability of programmers to work remotely, communicate, and share
information emphasized as more valuable skills today.
\end{abstract}

\cite{hoffmann2011}
In the 1970s, when Microsoft and Apple were founded,
programming was an art only a limited group of dedicated
enthusiasts actually knew how to perform properly. CPUs were rather slow,
personal computers had a very limited amount of memory, and monitors
were lo-res. To create something decent, a programmer had to
fight against actual hardware limitations.

In order to win in this war, programmers had to be both trained and talented in
computer science, a science that was at that time mostly about algorithms and data structures.
The first three volumes of the famous book \emph{The Art of Computer Programming} by Donald Knuth
were published in 1968-1973.
This programming bible earned a famous comment
from Bill Gates: ``It took incredible discipline, and several months, for me to read it.''

This comment demonstrates the reality that creating a simple piece of software was a complex engineering task,
even though software engineering only ``emerged as a discipline in its own right''
later on in 1980s, as Ian Sommerville
said in \emph{Software Engineering}.

Most programmers were calling themselves ``hackers.''
Even though in the early 80s this word,
according to Steven Levy's book \emph{Hackers},
``had acquired a specific and negative connotation.''
Since the 90s, this label has become
``a shibboleth that identifies one as a member of the tribe,''
as linguist Geoff Nunberg pointed out.

Being able to hack both the software and the hardware was a virtue for a long time.
However, the world of computer programming has changed dramatically in the last decade.

First of all, the \emph{cost} of computing power
gets cheaper every year.
For example, one gigabyte of computer memory cost about \$1,000 in 2000.
In 2018, it costs less than \$5. That is two hundred times cheaper in the span of only
eighteen years. The same is true for
hard drives, monitors, CPUs, and all
other hardware resources. As James Somers noticed in his analysis of industry problems in
\emph{The Coming Software Apocalypse}:
``Computers [have] doubled in power every 18 months for the last 40 years.''

Second, the growth of \emph{open source} is massive. The majority of software is available
for free now along with its source code, including operating systems,
graphic processors, compilers, editors, frameworks, cryptography
tools, and whatever else we can imagine. Programmers do not need to
write much code anymore; all they need to do in most cases is wire together
already available components.

\textcite{raymond1999}:
Good programmers know what to write. Great ones know what to rewrite (and reuse).
An important trait of the great ones is constructive laziness. They know that you get an A not for effort but for results, and that it's almost always easier to start from a good partial solution than from nothing at all.

Third, despite an increasing \emph{population} of programmers in the world, the field is still in
a deficit. In some European countries, the demand for highly-skilled
IT personnel is twice as high as their market supply of talent.
According to Iamexpat.nl,
the Netherlands expressed "a whopping 76\% of HR employees reported having difficulty
finding enough candidates with this qualification."

Fourth, programmers now work
\emph{remotely} instead of in offices or cubicles.
Thanks to the growth of high-speed internet,
conferencing software like Zoom and Skype, or
messaging tools like Slack
and
Telegram, and
distributed repository managers like
GitHub and Bitbucket,
along with many other innovations, remote work has become
more comfortable
than the alternative of working in the traditional office setting.

Finally, programmer \emph{salaries} have skyrocketed in the last few decades. In 2000, when one
gigabyte of memory still cost \$1,000, the average senior programmer
earned
around \$80,000 in Silicon Valley. In 2018, they are currently making
three times more,
while RAM is two hundred times cheaper.

Taking these five variables into account, it would appear that the skills required of
professional and successful programmers are drastically
different from the ones needed back in the 90s. The profession now requires less mathematics
and algorithms and instead emphasizes more skills under the umbrella term ``sociotech.''
Susan Long illustrates in her book, \emph{Socioanalytic Methods},
that the term ``sociotechnical systems"
was coined by Eric Trist et al. in the World War II era based on their work with
English coal miners at the Tavistock Institute in London. The term now seems more
suitable to the new skills and techniques modern programmers need.

They need to know how to \emph{communicate} with the open source community
to find the needed components,
to request features, and to learn bug fixes from
their developers. Moreover, they have to be ready to contribute to open
source software by submitting pull requests or even creating their own
programs. Those who used to work only with commercial and private software
will soon be far behind other programmers.

They have to know how to \emph{get help} outside of an office or even
a project team when working remotely and alone. Aside from
StackOverlow, that dominates the
Q\&A platform for programming market, there are documentation and
code repositories that a professional programmer must know how to navigate.
Those who previously only relied on colleagues and friends will now lose to those
who know how to learn from the entire internet.

Programmers have to know how to write \emph{maintainable} code that
other programmers will be able to easily understand. Since hiring personnel
grows more expensive every year, businesses emphasize the maintainability
of their code bases over developing exceptionally complex code. It is easier for them to buy a larger server
if the algorithm is not fast enough, rather than lose what previous
programmers created when a new team or a replacement shows up and fails
to understand how to modify the project. Because the cost of computers
continues to grow cheaper and the cost for employing programmers
continues to increase, maintainability continues to dominate the programming
landscape as the primary virtue of almost any software.
The end result is that these ``Hackers'' who spend their days writing
complex, cryptic code will soon find themselves out of the market.

Edsger W. Dijkstra's words---``Simplicity is a great virtue''---that
he uttered in 1984,
grow increasingly more valuable every year.

It seems that the future of programming rests less in math and more in
sociotech relationships between people.

\end{document}
